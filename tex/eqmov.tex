\section*{Maneouver Computations}
\addcontentsline{toc}{section}{Maneouver Computations}
Let us assume symmetric flight (no lateral aerodynamic force $Q=0$ nor lateral thrust force as $\nu=0$), the thrust generated by the engines is parallel to the \textit{x} wind axis (thus $\epsilon=0$),  and that the maneuver takes place flawlessly, this is in the vertical plane too; $\xi=\dot{\xi}=0$.\\
As the maneuver is a rather short, we will also assume that the fuel consumed i negligible and no mass fraction is lost; $\dot{m}=0$.\\
The motion equations follow:

\begin{equation*}
	\begin{cases}
	T\cos(\epsilon)\cos(\nu) - D -mg\sin\gamma-m\dot{V}=0\\
	T\cos(\epsilon)\sin(\nu) - Q+mg\cos\gamma\sin\mu+mV(\dot{\gamma}\sin\mu-\dot{\Xi}\cos\gamma\cos\mu)=0\\
	-T\sin\epsilon-L+mg\cos\gamma\cos\mu+mV(\dot{\gamma}\cos\mu-\dot{\Xi}\cos\gamma\sin\mu)=0\\
	\dot{x}_e=V\cos\gamma\cos\Xi\\
	\dot{y}_e=V\cos\gamma\sin\Xi\\
	\dot{x}_e=-V\sin\gamma
	\end{cases}
\end{equation*}

And for each of the phases that compose the maneuver, they can be simplified by substituting the flight conditions.

\begin{center}
\begin{tabular}{|l|c|c|c|}\hline
	& $\mu$ & $\gamma=\dot{\gamma}$ & $\xi=\dot{\xi}=\nu=\epsilon=Q$\\ \hline  \hline
	Cruise & 0 & 0 & 0 \\ \hline
	Semicirle & 0 & f(t) & 0 \\ \hline
	Inversion& f(t) & 0 & 0 \\ \hline
\end{tabular}
\end{center}

\subsection*{First phases - cruise}
\addcontentsline{toc}{subsection}{Maneouver Computations}
\begin{equation}
	\begin{cases}
		T - D -m\dot{V}=0\\
		0=0\\
		-L+mg=0\\
		\dot{x}_e=V\\
		\dot{y}_e=0\\
		\dot{x}_e=0
	\end{cases}
\end{equation}
There are three equations and four variables; $\alpha$, $\pi$ and V
Total degrees of freedom are:

\subsection*{Second phase - semicircle}
\addcontentsline{toc}{subsection}{Maneouver Computations}
\begin{equation}
	\begin{cases}
		T - D -mg\sin\gamma-m\dot{V}=0\\
		0=0\\
		-L+mg\cos\gamma+mV\dot{\gamma}=0\\
		\dot{x}_e=V\cos\gamma\\
		\dot{y}_e=0\\
		\dot{z}_e=-V\sin\gamma
	\end{cases}
\label{eq:semicircle}
\end{equation}
There are four equations and four variables: $\alpha$, $\pi$, V and $\gamma$. Timón de profundidad (elevator), palanca de gases (gas control lever or throttle).
Total degrees of freedom are:


\subsection*{Third phase - inversion}
\addcontentsline{toc}{subsection}{Maneouver Computations}
\begin{equation}
	\begin{cases}
		T - D -m\dot{V}=0\\
		mg\sin\mu=0\\
		-L-mg\cos\mu=0\\
		\dot{x}_e=V\\
		\dot{y}_e=0\\
		\dot{z}_e=0
	\end{cases}
\end{equation}
There are three equations and five variables; $\alpha$, $\pi$, V and $\mu$ Ailerons para mu.
Total degrees of freedom are:

\subsection*{Forth phase - cruise}
\addcontentsline{toc}{subsection}{Maneouver Computations}
\begin{equation}
	\begin{cases}
		T - D -m\dot{V}=0\\
		0=0\\
		-L+mg=0\\
		\dot{x}_e=V\\
		\dot{y}_e=0\\
		\dot{z}_e=0
	\end{cases}
\end{equation}
There are three equations and four variables; $\alpha$, $\pi$ and V.
Total degrees of freedom are: